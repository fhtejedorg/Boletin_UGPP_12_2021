\section{Seguimiento longitudinal}
\subssection{Dependientes}
ANDRYU: 
Máximo 4-5 paginas 

Contraste pareado :
Dejar la tabla 6, es solo sector privado. Esto ya esta en R 

Parrafo y explicación. 

Andryu: 
Tabla 7: 
Gráfico 6, Tabla 8 
Grafico 7 (presentación) 


Tabla 13. Distribución de los cotizantes de aportantes privados según la sección económica
y los contrastes entre junio 2019, junio 2020 y junio 2021

Tabla 14. Resumen de la variación mensual y anual de las cotizaciones en
dependientes de sector privado para junio 2021 en 12 principales departamentos.

Gráfico 11. Estructura de población de los cotizantes de junio 2020, junio 2021 y junio 2021
así como el índice de feminidad por grupos quinquenales de edad

\subssection{Independientes}
Rediseno de las tablas. Interntar ver los dos al mismo tiempo. 


Tabla 13. Total, cotizantes independientes que entran,
salen y permanecen entre junio 2021 y junio de 2021

Gráfico 13, , Porcentaje de cotizantes dependientes e independientes que salen y entran
para los meses de abril, mayo, mayo 2021 y junio 2021.

Tabla 15, Matriz de transición de los cotizantes independientes que permanecieron
en junio 2020 y junio 2021

27
Fuente: PILA, Cálculos UGPP
Gráfico 14. Porcentaje de las relaciones laborales que han aumentado y disminuido su rango
de cotización de los cotizantes independientes que han permanecido

Tabla 15. Distribución de los cotizantes independientes según la sección económica
y los contrastes entre junio 2019, junio 2020 y junio 2021

\begin{intemize}
\item 
\end{intemize}

